%%%%%%%%%%%%%%%%%%%%%%% file template.tex %%%%%%%%%%%%%%%%%%%%%%%%%
%
% This is a general template file for the LaTeX package SVJour3
% for Springer journals.          Springer Heidelberg 2010/09/16
%
% Copy it to a new file with a new name and use it as the basis
% for your article. Delete % signs as needed.
%
% This template includes a few options for different layouts and
% content for various journals. Please consult a previous issue of
% your journal as needed.
%
%%%%%%%%%%%%%%%%%%%%%%%%%%%%%%%%%%%%%%%%%%%%%%%%%%%%%%%%%%%%%%%%%%%
%
% First comes an example EPS file -- just ignore it and
% proceed on the \documentclass line
% your LaTeX will extract the file if required
\begin{filecontents*}{example.eps}
%!PS-Adobe-3.0 EPSF-3.0
%%BoundingBox: 19 19 221 221
%%CreationDate: Mon Sep 29 1997
%%Creator: programmed by hand (JK)
%%EndComments
gsave
newpath
  20 20 moveto
  20 220 lineto
  220 220 lineto
  220 20 lineto
closepath
2 setlinewidth
gsave
  .4 setgray fill
grestore
stroke
grestore
\end{filecontents*}
%
\RequirePackage{fix-cm}
%
%\documentclass{svjour3}                     % onecolumn (standard format)
%\documentclass[smallcondensed]{svjour3}     % onecolumn (ditto)
\documentclass[smallextended]{svjour3}       % onecolumn (second format)
%\documentclass[twocolumn]{svjour3}          % twocolumn
%
\smartqed  % flush right qed marks, e.g. at end of proof
%
\usepackage{graphicx}
%
\usepackage{hyperref}
%
\usepackage{mathptmx}      % use Times fonts if available on your TeX system
%
% insert here the call for the packages your document requires
\usepackage{latexsym}
% etc.
%
% please place your own definitions here and don't use \def but
\newcommand{\hospital}{`\textit{San Carlo di Nancy}' hospital }
%
% Insert the name of "your journal" with
\journalname{BISE}
%
\makeatletter
\newcommand{\setword}[2]{%
  \phantomsection
  #1\def\@currentlabel{\unexpanded{#1}}\label{#2}%
}
\makeatother
\begin{document}

\title{Applying process mining techniques in a real healthcare case study%\thanks{Grants or other notes
%about the article that should go on the front page should be
%placed here. General acknowledgments should be placed at the end of the article.}
}
%\subtitle{Do you have a subtitle?\\ If so, write it here}

%\titlerunning{Short form of title}        % if too long for running head

\author{M. Mecella \and
		F. Covino \and 
		A. Marrella\and \\
		S. Agostinelli}

%\authorrunning{Short form of author list} % if too long for running head

\institute{M. Mecella \at
			  Professor in Engineering in Computer Science at Sapienza University of Rome\\
              %Tel: +39 0677274028 \\
			  %Fax: +39 0677274002 \\
              \email{mecella@dis.uniroma1.it}           %  \\
%             \emph{Present address:} of F. Author  %  if needed
           \and
           A. Marrella \at
              Postdoctoral research fellow in Computer Science and Engineering at Sapienza University of Rome\\
              %Tel: +39-06-77274012 \\
			  %Tel: +39-06-77274013 \\
			  %Fax: +39-06-77274002 \\
              \email{a.marrella@gmail.com}           %  \\
%             \emph{Present address:} of F. Author  %  if needed
           \and
           F. Covino \at
           CMMI Business Analyst at ENAV\\
           \email{federico.covino@gmail.com}
           \and
           S. Agostinelli \at
           Graduated in Computer Engineering at Sapienza University of Rome\\
           \email{simone.agostinelli.sa@gmail.com}
}

\date{Received: date / Accepted: date}
% The correct dates will be entered by the editor


\maketitle

\begin{abstract}
%Insert your abstract here. Include keywords, PACS and mathematical
%subject classification numbers as needed.
The healthcare organizations are under increasing pressure to improve productivity, gain competitive advantage and reduce costs. For this reason, healthcare organizations, such as hospitals try to streamline their processes. In this paper we demonstrate the applicability of process mining in the healthcare domain, using a real case study of \hospital in Rome (GVM Group). We apply process mining techniques to obtain meaningful knowledge about the patient careflows from so-called event logs, obtained from raw data of hospital information systems. We analyzed these logs using the ProM framework from three different perspectives: the control flow perspective, the organizational perspective and the timing perspective. The results show that process mining can be used to provide new insights that facilitate the improvement of existing careflows.
\keywords{Process mining \and Healthcare \and ProM}
% \PACS{PACS code1 \and PACS code2 \and more}
% \subclass{MSC code1 \and MSC code2 \and more}
\end{abstract}
\clearpage
\section{Introduction}
Nowdays, the hospitals try to streamline their processes in order to deliver high quality care while at the same time improving revenues and reducing costs. More and more pressure is put on hospitals to work in the most efficient way as possible, whereas in the future, an increase in the demand for care is expected. A complicating factor is that healthcare is characterized by highly complex and extremely flexible patient care processes, also referred to as `\textit{careflows}'. In healthcare organisations, a wide range of processes with different characteristics and requirements are daily managed and executed. The delivery of complex care may involve several departments and organisations, and requires an active collaboration between different professionals and practitioners having heterogeneus skills. Healthcare is thus widely recognised as one of the most promising, yet challenging, domains for the adoption of process-oriented solutions. We demonstrates the applicability of process mining in the healthcare domain, using a real case study of \hospital in Rome (GVM Group). We apply process mining techniques to obtain meaningful knowledge about the patient careflows from so-called `\textit{event logs}' obtained from raw data of hospital information systems. Process mining aims at extracting process knowledge from that logs in order to discover, for example, both typical paths followed by particular groups of patients and strong collaboration between different hospitalization wards. We analyzed the different careflows both under the control flow perspective (emphasizing the differences between the process models obtained from different cluster of patients), the organizational perspective (looking at the social networks we were able to discover the relationship between the resources of the patient careflow) and the performance perspective (looking at the timing perspective of different activities performed by the patients we were able to discover bottlenecks in the patient careflow). In order to do so, we extracted the event logs from the raw datasets of \hospital and we analyzed them using \textit{ProM}: the process mining framework. The datasets in question are the following:
\begin{itemize}
\item \textit{Ambulatori} (outpatient clinic): each row stores the information about a single healthcare service.
\item \textit{Pronto soccorso} (emergency room): each row represents a single emergency room activity.
\item \textit{Ricoveri} (hospitalizations): each row represents a single hospitalization taken by a patient.
\end{itemize}
These three datasets contain raw data about patients treated in both year 2016 and May 2017 for which all the treatment activities have been recorded. We did not use any a-priori knowledge about the careflows of the patients of \hospital and did not have any process model at hand. The data analyzed are the standard ones of the National Health Service (Servizio Sanitario Nazionale) that the hospitals interchanged with the Regional Authorities (Enti Regione). Therefore the presented analysis can be replicated nationwide.%This paper is structured as follows: Section provides a detailed explanation of the tools used within ProM in order to derive process models, social networks and transition systems from relevant event logs. In Section we will emphasize the results obtained by applying process mining techniques. Section concludes the case study.
\newpage

%\label{adtool}
%Your text comes here. Separate text sections with
%\section{Section title}
%\label{sec:1}
%Text with citations \cite{RefB} and \cite{RefJ}.
%\subsection{Subsection title}
%\label{sec:2}
%as required. Don't forget to give each section
%and subsection a unique label (see Sect.~\ref{sec:1}).
%\paragraph{Paragraph headings} Use paragraph headings as needed.
%\begin{equation}
%a^2+b^2=c^2
%\end{equation}
\section{Analysis} \label{analysis}
In this section, we discuss the results obtained by applying process mining techniques (process discovery, social network analysis and performance analysis) on the datasets of \hospital already `\textit{converted}' to event logs: `\textit{Ambulatori}', `\textit{Pronto soccorso}' and `\textit{Ricoveri}'. The process mining techniques we used for discovering the process model of the totality of patients (blueprint) are: the \textit{Alpha Miner} algorithm, the \textit{Heuristic Miner} algorithm and the \textit{Inductive Miner} algorithm. In order to establish which process model out of three is representing better the behaviour observed in the event log, and therefore which mining algorithm is the best, we computed the four quality metrics replay fitness, precision, generalization and simplicity. ProM 6.7 allows to compute replay fitness, precision and generalization only. Therefore simplicity is calculated on the basis of this metric: \#$activities \;+\;$\#$splits \;+\; $\#$joins$. The grater is the value, the less is the simplicity of the process model. At the end, the best algorithm will be chosen on the basis of the just mentioned quality metrics. Since the inductive visual miner is high parametrisable ( i.e., we can set a-priori the replay fitness we would like to reach in order to obtain a process model in function of this fitness value) therefore it's the algorithm with the best compromise between the just mentioned quality metrics. For this reason, we picked the inductive visual miner as process discovery technique. After discovered the blueprint of all the three datasets, we have performed ad-hoc analysis for these set of clusters:
\begin{itemize}
\item foreign patients [\setword{1}{Word:foreign}] vs. italian patients [\setword{2}{Word:italian}]
\item patients with an age lower or equal than 25 years (resident in Latium)[\setword{3}{Word:young}] vs. patients with an age greater than 25 years (resident in Latium) [\setword{4}{Word:old}]
\end{itemize}
Only for `\textit{Ricoveri}' dataset we have performed a further in depth analysis for these specialist branches: \textit{chirurgia generale}, \textit{medicina generale} and \textit{ortopedia and traumatologia} looking at both pre-operative hospitalization time (the time a patient is waiting for the surgical intervention) and post-operative hospitalization time (the time a patient is waiting for being discharged). After did this phase of analysis and presented results, we performed a second ad-hoc analysis, with the goal of refining the overall procedure, on the following set of clusters:
\begin{itemize}
\item patients with an age lower or equal than 40 years (resident in Latium)[\setword{5}{Word:young2}] vs. patients with an age greater than 40 years (resident in Latium) [\setword{6}{Word:old2}]
\item patients performing health services in \textit{laboratorio analisi}, with medical prescription compiled by doctors of other hospitals, belonging to any specialist branches [\setword{7}{Word:lab00}]
\item patients coming from emergency room with hospitalization outcome [\setword{8}{Word:fromps}] vs. patients hospitalized not coming from emergency room [\setword{9}{Word:notfromps}], partitioned in specialist branches
\item patients hospitalized with reservation [\setword{10}{Word:reserv}], partitioned in different priority classes
\item patients hospitalized in \textit{chirurgia endocrina} with main diagnosis 24200/2410/2411 /2419/193 and surgical intervention 064/062 [\setword{11}{Word:endocrina}].
\end{itemize}
The third and last ad-hoc analysis involved the following set of clusters:
\begin{itemize}
\item patients performing health services in \textit{laboratorio radiologia}, with medical prescription compiled by doctors of other hospitals, belonging to any specialist branches [\setword{11}{Word:lab69}]
\item Integration of economic values regarding both the patients hospitalized coming from emergency room [\setword{12}{Word:fromps2}] and patients hospitalized not coming from emergency room [\setword{13}{Word:notfromps2}].
\item patients hospitalized resident in Latium [\setword{14}{Word:lazio}] vs. patients hospitalized resident in a region different from Latium  [\setword{15}{Word:NotLazio}]
\end{itemize}
In the last analysis phase we have also removed the duplicate data from both the `\textit{Pronto soccorso}' and `\textit{Ricoveri}' datasets. 
\newpage
\subsection{Ambulatori dataset} \label{analysis:1}
We first showed the blueprint of `\textit{Ambulatori}' in terms of process model, social network and transition system. In this way, we were able to see how patients within the dataset behave, the resources involved in the healthcare flow and the most time consuming activities. Then, we performed the ad-hoc analysis for the clusters already defined in the previous page.
\section{Conclusions} \label{conclusions}
% For one-column wide figures use
\begin{figure}
% Use the relevant command to insert your figure file.
% For example, with the graphicx package use
  \includegraphics{example.eps}
% figure caption is below the figure
\caption{Please write your figure caption here}
\label{fig:1}       % Give a unique label
\end{figure}
%
% For two-column wide figures use
\begin{figure*}
% Use the relevant command to insert your figure file.
% For example, with the graphicx package use
  \includegraphics[width=0.75\textwidth]{example.eps}
% figure caption is below the figure
\caption{Please write your figure caption here}
\label{fig:2}       % Give a unique label
\end{figure*}
%
% For tables use
\begin{table}
% table caption is above the table
\caption{Please write your table caption here}
\label{tab:1}       % Give a unique label
% For LaTeX tables use
\begin{tabular}{lll}
\hline\noalign{\smallskip}
first & second & third  \\
\noalign{\smallskip}\hline\noalign{\smallskip}
number & number & number \\
number & number & number \\
\noalign{\smallskip}\hline
\end{tabular}
\end{table}


%\begin{acknowledgements}
%If you'd like to thank anyone, place your comments here
%and remove the percent signs.
%\end{acknowledgements}

% BibTeX users please use one of
%\bibliographystyle{bise}      % Bise style
%\bibliographystyle{spbasic}      % basic style, author-year citations
%\bibliographystyle{spmpsci}      % mathematics and physical sciences
%\bibliographystyle{spphys}       % APS-like style for physics
%\bibliography{}   % name your BibTeX data base

% Non-BibTeX users please use
\begin{thebibliography}{}
%
% and use \bibitem to create references. Consult the Instructions
% for authors for reference list style.
%
\bibitem{RefJ}
% Format for Journal Reference
Author, Article title, Journal, Volume, page numbers (year)
% Format for books
\bibitem{RefB}
Author, Book title, page numbers. Publisher, place (year)
% etc
\end{thebibliography}

\end{document}
% end of file template.tex

